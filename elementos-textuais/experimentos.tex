\chapter{Experimentos}\label{chp:EXPERIMENTOS}

Este capítulo falará da solução em execução, ou seja, quais ferramentas escolhidas e seus motivos, como ele foi desenvolvido e como ele atuou em comparação aos trabalhos relacionados.

\section{Tecnologias Utilizadas}

Para o desenvolvimento da proposta, foi utilizado o \textit{framework} de \textit{frontend} Angular\footnote{\url{https://angular.io}} com Bootstrap\footnote{\url{https://getbootstrap.com}}, além do \textit{framework} AdonisJS\footnote{\url{https://adonisjs.com}}, construído em Node.js\footnote{\url{https://nodejs.org/en}}, no \textit{backend}. Para armazenamento de dados foi escolhido o SQLite\footnote{\url{https://www.sqlite.org/index.html}}, um sistema de gerenciamento de banco de dados relacional embutido, gratuito e de código aberto, que oferece integração com o Lucid, o ORM (\textit{Object-Relational Mapping}) utilizado pelo AdonisJS.

\subsection{Angular}
Angular é um \textit{framework} JavaScript de código aberto escrito em TypeScript e mantido pela Google. Ele é amplamente utilizado para desenvolver aplicações web baseadas em uma única página dinâmica, conhecidas também como SPAs (\textit{Single-Page Applications}). De acordo com \citeonline{ivanovs_2023}, o Angular é o segundo \textit{framework} \textit{frontend} mais utilizado no mundo desde 2016.

O Angular foi o \textit{framework} escolhido para construir o \textit{frontend} da aplicação por vários motivos. Uma de suas principais vantagens é sua arquitetura modular, que permite que o sistema seja dividido em módulos independentes. Essa abordagem facilita a organização do código e torna-o reutilizável. 

Outra vantagem é a sua arquitetura baseada em componentes, que permite criar interfaces interativas e reutilizáveis. Além disso, sua ampla comunidade de desenvolvedores e o suporte contínuo oferecido pela Google são ótimas vantagens de utilizar o \textit{framework}, por facilitar a busca por soluções para desafios encontrados durante o desenvolvimento.

\subsection{Bootstrap}

O Bootstrap é um \textit{framework} gratuito de código aberto utilizado para desenvolver interfaces responsivas. O \textit{framework} oferece uma biblioteca de estilos pré-definidos, com códigos escritos em HTML, CSS e JavaScript. Com as classes disponibilizadas pelo Bootstrap, podemos construir elementos como botões, menus e tabelas. Essas classes são responsivas, possibilitando que a aplicação seja usada em aparelhos com telas de diversos tamanhos.

Algumas vantagens de utilizar o Bootstrap são: responsividade, já que os elementos se adaptam a diferentes tipos de tela, o sistema de \textit{grid}, que divide o conteúdo da tela em linhas e colunas e facilita a criação da responsividade, e a compatibilidade com navegadores, já que o \textit{framework} é compatível com navegadores como Chrome, Edge, Firefox e Safari.

\subsection{Node.js}

O Node.js é um ambiente de execução de JavaScript do lado do servidor, o que significa que é possível executar aplicações Javascript fora do navegador. Ele utiliza o V8, conhecido também como \textit{Chrome’s V8 JavaScript engine}, um poderoso interpretador JavaScript desenvolvido pela Google que permite executar o código de forma assíncrona. O Node.js foi desenvolvido em 2009 por Ryan Dahl e, embora seja relativamente novo, é utilizado por grandes empresas como LinkedIn\footnote{\url{https://www.linkedin.com/}}, Netflix\footnote{\url{https://www.netflix.com/}}, Uber\footnote{\url{https://www.uber.com/}} e Trello\footnote{\url{https://trello.com/}}. \cite{brewster_2021}

Dentre as vantagens de utilizar o Node.js estão o modelo I/O não bloqueante e orientado a objetos que permite lidar com uma grande quantidade de chamadas sem gerar bloqueios ou gargalos. Além disso, ele oferece o NPM (\textit{Node Package Manager}), um gerenciador de pacotes que disponibiliza diversos pacotes de código aberto e reutilizável.


\subsection{AdonisJS}

O AdonisJS é um \textit{framework} opinado para Node.js criado em 2015 com o objetivo de fornecer uma estrutura sólida e completa para o desenvolvimento de aplicações web escaláveis.

Dentre as características notáveis do AdonisJS está seu padrão arquitetural MVC (\textit{Model-View-Controller}), que separa o código em modelos, visualizações e controladores, facilitando a manutenção e escalabilidade do código. Além disso, o AdonisJS oferece o Lucid ORM, um mapeador de objetos relacionais que oferece métodos prontos para simplificar a manipulação de banco de dados. O \textit{framework} também oferece recursos para autenticação, o que permite implementar métodos como login e cadastro de forma simples.

\subsection{SQLite}

O SQLite é um sistema de gerenciamento de banco de dados relacional embutido, gratuito e de código aberto. Diferente de sistemas como PostgreSQL\footnote{\url{https://www.postgresql.org}} e MySQL\footnote{\url{https://www.mysql.com}}, o SQLite funciona de forma independente e não necessita de um servidor.

Existem várias vantagens em usar o SQLite. Primeiramente, ele é altamente portável, e está disponível em plataformas como Windows, macOS, Linux, iOS e Android. Além disso, ele é fácil de configurar e não requer instalação, já que o banco de dados é armazenado em um único arquivo. Comparado a outros sistemas de banco de dados, o SQLite possui um tamanho de biblioteca relativamente pequeno, ocupando cerca de 250KB, enquanto servidores como o MySQL podem chegar a 600MB. \cite{estrella_2023}

No entando, é importante considerar algumas limitações do uso do SQLite. Dentre elas, a sua escalabilidade limitada, já que a biblioteca não é adequada para altas cargas. Além disso, o bloqueio do banco de dados em operações de gravação simultânea impacta o desempenho em cenários de intensa concorrência.

\subsection{Ambiente de Desenvolvimento}

Para o desenvolvimento da aplicação proposta utilizamos o Visual Studio Code\footnote{\url{https://code.visualstudio.com}}, um editor de código-fronte popular e versátil. De acordo com \citeonline{flow_2022}, o Visual Studio Code foi a ferramenta de edição de texto mais utilizada pelos desenvolvedores em 2022. Para gerenciamento do banco de dados, utilizamos o pacote MySQL\footnote{\url{https://dev.mysql.com/doc/visual-studio/en/visual-studio-code-editor.html}} do Visual Studio Code, que também oferece suporte para o SQLite. Para versionamento de código, a ferramenta escolhida foi o Git, uma ferramenta de versionamento utilizada por 93\% dos desenvolvedores. \cite{flow_2022}

%Duas citações? Deixa uma só no final?
%Quando usar cite? Quando usar citeonline?