\chapter{Experimentos}\label{chp:EXPERIMENTOS}

Este capítulo falará da solução em execução, ou seja, quais ferramentas escolhidas e seus motivos, como ele foi desenvolvido e como ele atuou em comparação aos trabalhos relacionados.

\section{Tecnologias Utilizadas}

Para o desenvolvimento da proposta, foi utilizado o \textit{framework} de \textit{frontend} Angular\footnote{\url{https://angular.io}} com Bootstrap\footnote{\url{https://getbootstrap.com}}, além do \textit{framework} AdonisJS\footnote{\url{https://adonisjs.com}}, construído em Node.js\footnote{\url{https://nodejs.org/en}}, no \textit{backend}. Para armazenamento de dados foi escolhido o SQLite\footnote{\url{https://www.sqlite.org/index.html}}, um sistema de gerenciamento de banco de dados relacional embutido, gratuito e de código aberto, que oferece integração com o Lucid, o ORM (\textit{Object-Relational Mapping}) utilizado pelo AdonisJS.

\subsection{Angular}
Angular é um \textit{framework} JavaScript de código aberto escrito em TypeScript e mantido pela Google. Ele é amplamente utilizado para desenvolver aplicações web baseadas em uma única página dinâmica, conhecidas também como SPAs (\textit{Single-Page Applications}).

O Angular foi o \textit{framework} escolhido para construir o \textit{frontend} da aplicação por vários motivos. Uma de suas principais vantagens é sua arquitetura modular, que permite que o sistema seja dividido em módulos independentes. Essa abordagem facilita a organização do código e torna-o reutilizável. 

Outra vantagem é a sua arquitetura baseada em componentes, que permite criar interfaces interativas e reutilizáveis. 

Além disso, sua ampla comunidade de desenvolvedores e o suporte contínuo oferecido pela Google são ótimas vantagens de utilizar o \textit{framework}, por facilitar a busca por soluções para desafios encontrados durante o desenvolvimento.

%Pesquisa sobre Angular ser o 2 framework mais utilizado
%https://stackdiary.com/front-end-frameworks/
%como citar?

\subsection{Bootstrap}

O Bootstrap é um \textit{framework} gratuito de código aberto utilizado para desenvolver interfaces responsivas. O \textit{framework} oferece uma biblioteca de estilos pré-definidos, com códigos escritos em HTML, CSS e JavaScript. Com as classes disponibilizadas pelo Bootstrap, podemos construir elementos como botões, menus e tabelas. Essas classes são responsivas, possibilitando que a aplicação seja usada em aparelhos com telas de diversos tamanhos.

Algumas vantagens de utilizar o Bootstrap são: responsividade, já que os elementos se adaptam a diferentes tipos de tela, o sistema de \textit{grid}, que divide o conteúdo da tela em linhas e colunas e facilita a criação da responsividade, e a compatibilidade com navegadores, já que o \textit{framework} é compatível com navegadores como Chrome, Edge, Firefox e Safari.

\subsection{Node.js}

\subsection{AdonisJS}

\subsection{SQLite}

\subsection{Ambiente de Desenvolvimento}