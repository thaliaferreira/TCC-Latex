\chapter{Proposta}\label{chp:PROPOSTA}

No Brasil, desde 2004 por meio da Lei n° 10.861 as instituições de ensino superior devem estar de acordo com as normas do Sistema Nacional de Avaliação da Educação Superior (SINAES)
cujo objetivo é contribuir para a melhoria contínua dos cursos e instituições avaliando 
os aspectos relacionados ao ensino, pesquisa, extensão, do corpo docente, da gestão institucional, bem como a responsabilidade social e infraestrutura.
Contudo, a avaliação não é um processo meramente técnico e seu sucesso depende, em grande parte, do reconhecimento da legitimidade dos responsáveis por sua realização (Dias Sobrinho, 2000). 

Para que a instituição esteja preparada para enfrentar desafios, é fundamental que seus membros tenham ciência de sua realidade, virtudes, capacidades e limitações.
Segundo Nelson de Abreu Júnio (2009), os processos avaliativos precisam envolver o maior número de participantes, tanto na construção de seu projeto quanto na análise e no uso dos resultados, contribuindo para o desenvolvimento humano. Desta forma, e com base no diagnóstico de suas condições, a tomada de decisões poderá ser feita de maneira ética.


%%%%%%%%%%%%%%%%%%%%%%%%%%%%%%%%%%%%%%%%%%%%%%%%%%%%%%%%%%%%%%%%%%%%%
% - Importâncida da avaliação institucional no que diz respeito de detectar demandas/melhorias/pontos fracos
% - SINAES como medidor de qualidade de ensino superior
% - Avaliação interna abrange situações/necessidades mais específicas
% - 
%

% Esse capítulo será responsável por explicar como será a sua solução.
% Ele deverá explicar o problema em que a sua solução irá resolver.
% Nele irá conter COMO deverá ser a sua solução, ou seja, neste momento você não está preocupado com a implementação ou ferramentas.
% Aqui será relatado o problema e a sua proposta. 
% Nela será incluída a modelagem da solução, sua arquitetura e tudo o que for necessário para que o leitor consiga entender COMO será a solução e como ela resolverá o problema relatado.

% Além disso, uma parte fundamental, é tratar de trabalhos relacionados. Dependendo da forma de escrita, o trabalho relacionado pode estar explicado no capítulo de Fundamentação ou ser uma seção dentro da proposta antes de entrar na proposta em si.