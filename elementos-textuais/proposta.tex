\chapter{Proposta}\label{chp:PROPOSTA}

No Brasil, desde 2004 por meio da Lei n° 10.861 as instituições de ensino superior devem estar de acordo com as normas do Sistema Nacional de Avaliação da Educação Superior (SINAES)
cujo objetivo é contribuir para a melhoria contínua dos cursos e instituições avaliando 
os aspectos relacionados ao ensino, pesquisa, extensão, do corpo docente, da gestão institucional, bem como a responsabilidade social e infraestrutura.

Para que a instituição esteja preparada para enfrentar desafios, é fundamental que seus membros tenham ciência de sua realidade, virtudes, capacidades e limitações.
Segundo Nelson de Abreu Júnio[2] (2009), os processos avaliativos precisam envolver o maior número de participantes, tanto na construção de seu projeto quanto na análise e no uso dos resultados, contribuindo para o desenvolvimento humano. Desta forma, e com base no diagnóstico de suas condições, a tomada de decisões poderá ser feita de maneira ética.


\section{Proposta}
O intuito desse trabalho é desenvolver um sistema web onde os alunos da Universidade Federal Rural do Rio de Janeiro possam avaliar as disciplinas ofertadas ao longo do período. O sistema permite que os aluno acessem um questionário de avaliação para cada disciplina em que ele estava matriculado. Esse questionário conta com perguntas que podem medir a qualidade da didática do professor, a ementa da disciplina, as avaliações, entre outros.

O sistema conta com uma tela de cadastro que pode ser usada tanto por alunos e professores da universidade e uma tela de login. Caso o usuário logado seja um aluno, ele poderá visualizar uma lista das disciplinas em que ele estava matriculado. O aluno pode avaliar cada disciplina dessa lista somente uma vez. Caso o usuário logado seja um professor, ele terá acesso a uma lista de disciplinas lecionadas por ele. Cada disciplina contará com um relatório com a média de avaliações recebidas. Caso o usuário seja um administrador, ele poderá acessar todas as disciplinas do curso de Ciência da Computação da UFRRJ, e visualizar o relatório de avaliações para cada uma delas.

Administradores do sistema poderão cadastrar períodos e disciplinas, e associar os professores a suas respectivas disciplinas. 

\section{Requisitos de sistema}
Os requisitos de sistema são descrições sobre o que o sistema deve ser capaz de fazer para cumprir com as regras de negócios. Eles podem ser divididos em requisitos funcionais e não funcionais. Os requisitos funcionais são aqueles que o sistema deve ter para atender a necessidade do usuário final. Já os não funcionais devem apresentar as limitações do sistema desenvolvido.

Ao satisfazer todos os requisitos de sistema, podemos garantir que o software é confiável e funcional.

-Visualizar lista de disciplina

-Avaliar disciplina

-Cadastrar/editar/excluir disciplina/período

-Visualizar relatórios

\begin{table}[]
    \centering
    \caption{Requisitos do Sistema}
    \label{tab:tabela_requisios_do_sistema}
    \resizebox{\textwidth}{!}{%
    \rowcolors{1}{}{lightlightgray}
    \begin{tabular}{ccc}
    \hline
    \textbf{ID} & \textbf{Nome} & \textbf{Descrição} \\ \hline
    \textbf{RF01} & Cadastrar disciplina & \begin{tabular}[c]{@{}c@{}}O sistema permiritá que o usuário inclua disciplina\\ informando NOME, TURNO, PROFESSOR, PERÍODO LETIVO\end{tabular} \\
    \textbf{RF02} & Editar disciplina & \begin{tabular}[c]{@{}c@{}}O sistema permitirá que o usuário edite \\o NOME, TURNO, PROFESSOR e PERÍODO LETIVO de uma disciplina \end{tabular} \\
    \textbf{RF03} & Excluir disciplina & O sistema permitirá que o usuário exclua uma disciplina \\
    \textbf{RF04} & Listar disciplinas & \begin{tabular}[c]{@{}c@{}}O sistema permitirá a visualização de uma disciplina\\ (NOME, TURNO, PROFESSOR)\end{tabular} \\
    \textbf{RF05} & Cadastrar período & O sistema permitirá que o usuário inclua um período letivo (ANO) \\
    \textbf{RF06} & Editar período & \begin{tabular}[c]{@{}c@{}}O sistema permitirá que o usuário edite o ano \\ de um período letivo cadastrado no sistema\end{tabular} \\
    \textbf{RF07} & Excluir período & O sistema permitirá que o usuário exclua um período letivo \\
    \textbf{RF08} & Visualizar relatório & \begin{tabular}[c]{@{}c@{}}O sistema permitirá a visualização de \\ períodos letivos cadastrados no sistema\end{tabular}
    \end{tabular}%
    }
\end{table}

\section{Casos de Uso}
Os casos de uso são uma metodologia com objetivo de documentar as principais funcionalidades do sistema do ponto de vista do usuário final. Eles consistem de possíveis sequências de interações entre o usuário e o sistema.

-Cadastro

-Login

-Visualizar disciplinas

-Avaliar disciplina

-Cadastrar disciplina

-Cadastrar período

-Visualizar relatórios

\begin{table}[]
    \centering
    \caption{Casos de Uso}
    \label{tab:tabela_casos_de_uso}
    \resizebox{\textwidth}{!}{%
    \rowcolors{1}{}{lightlightgray}
    \begin{tabular}{cccc}
    \hline
    \textbf{ID} & \textbf{Nome} & \textbf{Agente} & \textbf{Descrição} \\ \hline
    \textbf{UC01} & Efetuar login & \begin{tabular}[c]{@{}c@{}}Aluno, Coordenador\\ ou Professor\end{tabular} & \begin{tabular}[c]{@{}c@{}}Autenticação de usuários cadastrados no sistema,\\  permitindo a realização de operações restritas à cada perfil\end{tabular} \\
    \textbf{UC01} & Efetuar login & \begin{tabular}[c]{@{}c@{}}Aluno, Coordenador\\ ou Professor\end{tabular} & \begin{tabular}[c]{@{}c@{}}Autenticação de usuários cadastrados no sistema,\\ permitindo a realização de operações restritas à cada perfil\end{tabular} \\
    \textbf{UC02} & Cadatrar usuário & Coordenador & \begin{tabular}[c]{@{}c@{}}Inclusão no sistema usuários para  executar as operações de acordo \\ com o perfil (Aluno ou Professor)\end{tabular} \\
    \textbf{UC03} & Visualizar disciplina & Aluno ou Professor & Visualização das informações da disciplina: Ano letivo, professor \\
    \textbf{UC04} & Avaliar disciplina & Aluno & Avaliação da disciplina cursada no ano letivo em questão \\
    \textbf{UC05} & Cadastrar disciplina & Coordenador & \begin{tabular}[c]{@{}c@{}}Cadastro de uma disciplina no sistema informando NOME, PERÍODO LETIVO, \\ TURNO e PROFESSOR\end{tabular} \\
    \textbf{UC06} & Cadastrar período & Professor & Cadastro de ano letivo no sistema (ANO) \\
    \textbf{UC07} & Visualizar relatório & Coordenador ou Professor & Visualização das avaliações feitas pelos alunos da disciplina
    \end{tabular}%
    }
\end{table}


\section{Trabalhos Relacionados}
Neste tópico serão apresentados alguns sistemas de avaliação aplicados por universidades e trabalhos acerca da avaliação institucional no ensino superior.
De uma maneira mais geral, temos o \href{https://www.ratemyprofessors.com/}{Rate my Professors}, que foi criado
com a motivação de ajudar outros estudantes universitários a tomar decisões informadas ao escolher suas aulas e professores, acreditando que as opiniões dos alunos poderiam ajudar futuros alunos, tornando a seleção de aulas e professores mais transparente e informada.

A Universidade Federal Fluminense (UFF) conta com o Sistema de Avaliação Institucional (SAI) desde 2003. Uma ferramenta
crucial para a gestão de qualidade na univerdidade. Além de oferecer as avaliações de disciplinas e do corpo docente pelos estudantes, abrange também questões de infraestrutura, pesquisa de opinião e indicadores de satisfação.

Falar sobre universidades que aplicam avaliações internas:
Exemplo: 
Universidad de Buenos Aires (UBA)
Universidade de São Paulo (USP) = Avaliação Institucional da USP
Universidade Federal FLuminense (UFF) = Avaliação Institucional da UFF
%%%%%%%%%%%%%%%%%%%%%%%%%%%%%%%%%%%%%%%%%%%%%%%%%%%%%%%%%%%%%%%%%%%%%
% [1] - Cad. Cedes, Campinas vol. 29, n. 78, p. 257-269, maio/ago. 2009 257 Disponível em <http://www.cedes.unicamp.br>
% [2] - 
% - Importâncida da avaliação institucional no que diz respeito de detectar demandas/melhorias/pontos fracos
% - SINAES como medidor de qualidade de ensino superior
% - Avaliação interna abrange situações/necessidades mais específicas
% - 
%

% Esse capítulo será responsável por explicar como será a sua solução.
% Ele deverá explicar o problema em que a sua solução irá resolver.
% Nele irá conter COMO deverá ser a sua solução, ou seja, neste momento você não está preocupado com a implementação ou ferramentas.
% Aqui será relatado o problema e a sua proposta. 
% Nela será incluída a modelagem da solução, sua arquitetura e tudo o que for necessário para que o leitor consiga entender COMO será a solução e como ela resolverá o problema relatado.

% Além disso, uma parte fundamental, é tratar de trabalhos relacionados. Dependendo da forma de escrita, o trabalho relacionado pode estar explicado no capítulo de Fundamentação ou ser uma seção dentro da proposta antes de entrar na proposta em si.