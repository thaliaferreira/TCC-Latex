\chapter{Proposta}\label{chp:PROPOSTA}

No Brasil, desde 2004 por meio da Lei n° 10.861 as instituições de ensino superior devem estar de acordo com as normas do Sistema Nacional de Avaliação da Educação Superior (SINAES)
cujo objetivo é contribuir para a melhoria contínua dos cursos e instituições avaliando 
os aspectos relacionados ao ensino, pesquisa, extensão, do corpo docente, da gestão institucional, bem como a responsabilidade social e infraestrutura.
Contudo, a avaliação não é um processo meramente técnico e seu sucesso depende, em grande parte, do reconhecimento da legitimidade dos responsáveis por sua realização (Dias Sobrinho, 2000). 

Para que a instituição esteja preparada para enfrentar desafios, é fundamental que seus membros tenham ciência de sua realidade, virtudes, capacidades e limitações.
Segundo Nelson de Abreu Júnio (2009), os processos avaliativos precisam envolver o maior número de participantes, tanto na construção de seu projeto quanto na análise e no uso dos resultados, contribuindo para o desenvolvimento humano. Desta forma, e com base no diagnóstico de suas condições, a tomada de decisões poderá ser feita de maneira ética.


\section{Proposta}
O intuito desse trabalho é desenvolver um sistema web onde os alunos da Universidade Federal Rural do Rio de Janeiro possam avaliar as disciplinas ofertadas ao longo do período. O sistema permite que os aluno acessem um questionário de avaliação para cada disciplina em que ele estava matriculado. Esse questionário conta com perguntas que podem medir a qualidade da didática do professor, a ementa da disciplina, as avaliações, entre outros.

O sistema conta com uma tela de cadastro que pode ser usada tanto por alunos e professores da universidade e uma tela de login. Caso o usuário logado seja um aluno, ele poderá visualizar uma lista das disciplinas em que ele estava matriculado. O aluno pode avaliar cada disciplina dessa lista somente uma vez. Caso o usuário logado seja um professor, ele terá acesso a uma lista de disciplinas lecionadas por ele. Cada disciplina contará com um relatório com a média de avaliações recebidas. Caso o usuário seja um administrador, ele poderá acessar todas as disciplinas do curso de Ciência da Computação da UFRRJ, e visualizar o relatório de avaliações para cada uma delas.

Administradores do sistema poderão cadastrar períodos e disciplinas, e associar os professores a suas respectivas disciplinas. 

\section{Requisitos de sistema}
Os requisitos de sistema são descrições sobre o que o sistema deve ser capaz de fazer para cumprir com as regras de negócios. Eles podem ser divididos em requisitos funcionais e não funcionais. Os requisitos funcionais são aqueles que o sistema deve ter para atender a necessidade do usuário final. Já os não funcionais devem apresentar as limitações do sistema desenvolvido.

Ao satisfazer todos os requisitos de sistema, podemos garantir que o software é confiável e funcional. 

-Visualizar lista de disciplina

-Avaliar disciplina

-Cadastrar/editar/excluir disciplina/período

-Visualizar relatórios

\section{Casos de Uso}
Os casos de uso são uma metodologia com objetivo de documentar as principais funcionalidades do sistema do ponto de vista do usuário final. Eles consistem de possíveis sequências de interações entre o usuário e o sistema.

-Cadastro

-Login

-Visualizar disciplinas

-Avaliar disciplina

-Cadastrar disciplina

-Cadastrar período

-Visualizar relatórios

%%%%%%%%%%%%%%%%%%%%%%%%%%%%%%%%%%%%%%%%%%%%%%%%%%%%%%%%%%%%%%%%%%%%%
% - Importâncida da avaliação institucional no que diz respeito de detectar demandas/melhorias/pontos fracos
% - SINAES como medidor de qualidade de ensino superior
% - Avaliação interna abrange situações/necessidades mais específicas
% - 
%

% Esse capítulo será responsável por explicar como será a sua solução.
% Ele deverá explicar o problema em que a sua solução irá resolver.
% Nele irá conter COMO deverá ser a sua solução, ou seja, neste momento você não está preocupado com a implementação ou ferramentas.
% Aqui será relatado o problema e a sua proposta. 
% Nela será incluída a modelagem da solução, sua arquitetura e tudo o que for necessário para que o leitor consiga entender COMO será a solução e como ela resolverá o problema relatado.

% Além disso, uma parte fundamental, é tratar de trabalhos relacionados. Dependendo da forma de escrita, o trabalho relacionado pode estar explicado no capítulo de Fundamentação ou ser uma seção dentro da proposta antes de entrar na proposta em si.